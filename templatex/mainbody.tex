%=================================================================
\section{Introduction}\label{sec-intro}


\todo{Narrow down to a topic; Dig a hole; Fill the hole}



\gangli{``narrow in on topic'' reminds you 
that readers and reviewers only know that this is a AI or HTM research paper (and maybe have read the title/abstract). 
You need to help them figure out what topic and area of research paper this is. 
You _don't_ need to wax poetic about the topic's importance.}

\gangli{`dig a hole'' reminds you that 
you need to convince the reader that there's a problem with the state of the world. 
Prior work may exist but it's either missing something important or there's a missing opportunity. 
The reader should be drooling for a bright future just out of reach.}

\gangli{
``fill the hole'' reminds you to show the reader 
how and why the paper they're reading will fix these problems and deliver us into a better place. 
You don't need a whirlwind summary of the technical details, 
but you need readers convinced (and in a good mood) to keep reading.}



\todo{The importance of the area}
\blindtext


\todo{The problems faced by most current methods}
\blindtext

\todo{What can be addressed by existing methods; Why those problems are challenges to existing methods?}
\blindtext

\todo{What provides the motivation of this work? What are the research issues? What is the rationale of this work? }
\blindtext

\todo{What we have done and what are the contributions.}
\blindtext



Test citation~\cite{BL12J01}. 
\begin{JournalOnly}
and~\citep{BJL11J01} or~\citet{BJL11J01}.
\end{JournalOnly}

This is for~\cref{tbl:overall-experiments}, 
\todo[fancyline]{Testing.}
and this is for~\cref{sec-conclusions}.
\todo[noline]{A note with no line back to the text.}%
\gangli{This is comment from Gang.}
\qwu{Response from QW}

Number:
\num{123}.
\numlist{10;30;50;70},
\numrange{10}{30},
\SIlist{10;30;45}{\metre},
and
\SI{10}{\percent}

\missingfigure[figcolor=white]{Testing figcolor}


\begin{ConferenceOnly}
We have \SI{10}{\hertz},
\si{\kilogram\metre\per\second},
the range: \SIrange{10}{100}{\hertz}.
$\nicefrac[]{1}{2}$.

\missingfigure{Make a sketch of the structure of a trebuchet.}

\end{ConferenceOnly}


For~\cref{eq:test},
as shown below:

\begin{equation}\label{eq:test}
a = b \times \sqrt{ab}
\end{equation}

\blindmathpaper

\section{Preliminaries} \label{sec-preliminaries}

\blindtext

\gliMarker  %TODO: GLi Here


\section{Method} \label{sec-method}

\blindtext
\blindlist{itemize}[3]
\blinditemize
\blindenumerate

\blindmathtrue
\blindmathfalse
\blinddescription

\qwuMarker %TODO: QWu Here

\section{Experiment and Analysis} \label{sec-experiment}


\begin{table}  \centering
  \caption{Precision Comparison on Event Detection Methods}
  \label{tbl:overall-experiments}
  \begin{tabular}{cccc}
\toprule
    % after \\: \hline or \cline{col1-col2} \cline{col3-col4} ...
    & OR Event Detection & AC Event Detection & TC Event Detection \\
\midrule
    precision & 0.83 & 0.69 & 0.46 \\
    recall & 0.68 & 0.48 & 0.36 \\
    F-score & 0.747 & 0.57 & 0.4 \\
\bottomrule
\end{tabular}
\end{table}


\section{Conclusions} \label{sec-conclusions}

\blindtext

\section*{Acknowledgement}

\lipsum[1]


The authors would like to thank \ldots

